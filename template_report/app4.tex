\chapter{About references}\label{app:refs}
Your report will be put together in your own style, mostly using your
own words. Much of it will be standard material that you've read and
digested, but you may have a fresh example, application or calculation
that you've done yourself.
\par
You must make clear what's not your own work by referring suitably
to your sources --- books or articles or web-pages, etc. Not to do so
can count as plagiarism, which is cheating.
\par
This appendix aims to amplify the advice given in the
Library's guide \cite{BR}, which you should read.
%----------------------------------------
\section{Organisation}
Follow the style of this template --- that is, put an orderly list of
your sources (\Quote{The Bibliography}) at the end of the main text,
and refer to (\Quote{cite}) items on the list by suitable codes placed
appropriately in the report's text.
\par
For your bibliography the golden rule is that each listed item must give
enough detail to allow readers to follow it up themselves and find the
precise part of the book, article, web-site or whatever without
ambiguity or delay.
\par
For instance it's no use saying just \Quote{The Times newspaper} unless
you also give the date and page, or just \Quote{Wikipedia} unless you
give the complete URL of the specific page, \dots\ and so on.
\par
It's always better to give slightly too much information --- e.g. you
might include the ISBN of a book too. A good level of information is
exemplified here and recommended in the Library guide \cite{BR}.
\par
If you need to cite different parts of e.g. a book for different things,
then list it (say AB) in the bibliography and give the different
citations as [AB,~page~32] and [AB,~Sec.~4.7] and so on. The \LaTeX\
\verb+\cite+ command allows for this (look it up!).
\par
Now you see why a bibliography list is better than lots of footnotes
--- it neatly allows such multiple citations. Too many footnotes make a
mess.
\begin{itemize}\item This bibliography uses short letter-code
keys in alphabetical order. It's one of several standard possibilities
\cite{BR} and is often preferred for economy and because the letter
codes\footnote{Note the punctuation of \lq letter-code' and \lq letter
codes' here.} can be chosen to be helpful mnemonics. Numerical codes,
for instance, can't.
\item Long lines (with URLs usually \cite{IM,PS}) may need to be split.
The \LaTeX\ hack used here to manage such line-breaks may not appeal to
everyone.\end{itemize}
%---------------------------------
\section{Doing it}
What things need a reference? The answer is --- anything you didn't work
out or invent yourself, but took from someone or somewhere else. You may
do that, so long as you say so \textit{and also make it clear you
understand it}. Don't copy blindly! Here are some
examples.\begin{itemize}
\item \Quote{the following explanation is taken from [LH]} --- if you've
copied word-for-word from the article by Laurel \& Hardy, which you
list as item LH in the bibliography. Direct quotation should be used
sparingly, and the text emphasised --- perhaps by use of the
\texttt{quotation} environment in \LaTeX.\par
Don't be tempted to copy from a book, or \texttt{ctrl-C/ctrl-V} from the
web, \textit{without} clearly admitting it. It's easy to detect, and
it's suicide.
\item \Quote{the following proof is in many textbooks, \eg
[LH,~page~16]} --- if it is, and you can't think of a different proof,
except perhaps for notation.
\item \Quote{the material in this section is based on Sec.~2.6 of [CL]
and Chap.~3 of [SH]} --- if you've combined into your own words the
account by Cagney \& Lacey with that by Starsky \& Hutch.
 \item \Quote{the following proof is adapted from Chap.~4 of [SJ]} ---
if (say) you've filled in the gaps in Smith \& Jones' proof, or perhaps
changed it from the case of general $n$ to your case $n=2$.
\item \Quote{these calculations were done with \textsl{Matlab} [MAT]
using the m-file listed in App.~\ref{app:programs}} --- where the
reference MAT is to the \textsl{MathWorks} website.
\item \Quote{these calculations were done with the TISEAN package [TIS]}
--- if you've downloaded this specialised package from a web-page whose
URL (and author's name, if available) is item TIS in the bibliography.
\item \Quote{the graph is taken from Morecambe \& Wise [MW, page 16]}
--- if you've scanned the figure from their book. Similarly, if you've
downloaded a \texttt{.jpg} file from a web-site, then give in the
bibliography the URL and author (if known). Such citations could go
either in the figure caption or in the associated text.
\item \Quote{the data are taken from [XY]} --- where XY gives the
source of the numbers you've analysed. This might be a journal or
an online databank, \etc. Generally the numbers themselves should be
left out
--- a summary table or a graph or two (plotted with \textsl{R} or
\textsl{Matlab} or \textsl{Maple}) is often enough. Occasionally raw
data might be supplied separately --- e.g. on a CDrom.
\end{itemize}
Often your supervisor or someone else gives you something --- such
as a set of data, or a useful \textsl{Maple} worksheet, or help with a
proof --- when the appropriate form of bibliography item is
\Quote{Dr~I~Newton, private communication, April 2007}. Similarly for
printed course material --- \Quote{Dr~I~Newton, lecture notes
for module MATH5033, Durham University, Epiphany Term 2007.}
\par
Generally, don't refer to things you haven't read. Textbooks may well
cite the Serbian-language journal where the result you want was
originally published. But you are not in the business of ascribing
credit for discovery. In \textit{your} report \textit{you} must cite the
book where \textit{you} actually got it from and not give any possibly
false impression that you fluently read technical Serbian.
\par
Likewise, you may want to quote from Euclid or Archimedes.
But cite the place where you found the English words with something like
\Quote{Archimedes said, \lq Eureka!\rq\ (as quoted by [AB])}.
\par
Summarising --- tell the truth (where you got it from), the whole truth
(give full details), and nothing but the truth (you don't read Serbian).
\par
Finally, if in doubt --- ask your supervisor.
