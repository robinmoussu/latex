\documentclass[12pt,a4paper]{moderncv}
%\moderncvtheme[orange]{custom}
\moderncvstyle{classic}
\moderncvcolor{orange}

\usepackage[utf8]{inputenc}
\usepackage[left=2.1cm, right=2.1cm, top=1.1cm, bottom=1.3cm]{geometry}
\usepackage{relsize}
\usepackage{bold-extra}
\usepackage{lmodern}
\usepackage{calc}

%\fontsize{13}{15.6}\selectfont

% \input{style.sty}

%%%%%%%%%%%%%%%%%%%%%%%%%%%%%%%%%%%%%%%%%%%%%%%%%%%%%%%%%%

% \setlength{\hintscolumnwidth}{22mm}

%\newcommand{\cvformation}{\cventry}
 \newcommand{\cvformation}[7][.25em]{%
  \cvitem[#1]{#2}{%
    {\textbf{#3}}%
    \ifthenelse{\equal{#4}{}}{}{, {\textit{#4}}}%
    \ifthenelse{\equal{#5}{}}{}{, #5}%
    \ifthenelse{\equal{#6}{}}{}{, #6}%
    \ifx&#7&%
      \else{\newline#7}\fi}
}
\newcommand{\cvstage}[7][.25em]{%
  \cvitem[#1]{#2}{%
    {\textbf{#7}}%
    \ifx&#6&%
      \else{\hfill\small#6}\fi
    \newline%
    \textit{#4}%
    \ifthenelse{\equal{#5}{}}{}{, #5}%
  }
}

\newcommand{\cvcomp}[7][.25em]{%
   \cvitem[#1]{#2}{%
     {\textbf{#3}}%
 %
     \ifthenelse{\equal{#4}{}}{%
       \ifthenelse{\equal{#5}{}}{%
         \ifthenelse{\equal{#6}{}}{%
         }{, #6}%
       }{, #5}%
     }{, \slshape#4}%
 %
     \ifthenelse{\equal{#5}{}}{}{, #5}%
     \ifthenelse{\equal{#6}{}}{}{, #6}%
     \ifx&#7&%
       \else{\hfill{}#7}\fi}
 }

\newcommand{\cvcomptwo}[7][.25em]{%
   \cvitem[#1]{#2}{%
     {\textbf{#3}}%
 %
     \ifthenelse{\equal{#4}{}}{%
       \ifthenelse{\equal{#5}{}}{%
         \ifthenelse{\equal{#6}{}}{%
         }{, #6}%
       }{, #5}%
     }{, \slshape#4}%
 %
     \ifthenelse{\equal{#5}{}}{}{, #5}%
     \ifthenelse{\equal{#6}{}}{}{, #6}%
     \ifx&#7&%
       \else{\hfill{}\break{}#7}\fi}
 }


\newlength{\cvitemrightoffset}
\setlength{\cvitemrightoffset}{1cm}
\newcommand{\cventryright}[3][.25em]{%
  \recomputecvlengths%
  \begin{tabular}{@{}p{\cvitemrightoffset + \hintscolumnwidth}@{\hspace{\separatorcolumnwidth}}p{\maincolumnwidth - \cvitemrightoffset}@{}}%
 	  \raggedleft\hintstyle{#2} &\raggedleft{#3}%
  \end{tabular}%
  \par\addvspace{#1}}

% calculate age
\newcounter{age}
\setcounter{age}{\the\year}
\addtocounter{age}{-1992}

%%%%%%%%%%%%%%%%%%%%%%%%%%%%%%%%%%%%%%%%%%%%%%%%%%%%%%%%%%

%C plus plus
\newcommand\cpp{C\nolinebreak[4]\hspace{-.05em}\raisebox{.4ex}{\relsize{-3}{\textbf{++}}}}
\newcommand\pic{\textsc{Pic}}
\newcommand\stm{\textsc{Stm32}}
\newcommand\uml{\textsc{uml}}
\newcommand\labview{Lab\textsc{view}}

%%%%%%%%%%%%%%%%%%%%%%%%%%%%%%%%%%%%%%%%%%%%%%%%%%%%%%%%%%

\newcommand{\xx}[1]{\textbf{\large#1}}

\ifdefined\isenglish
  \newcommand{\fr}[1]{}
  \newcommand{\en}[1]{#1}
\else
  \newcommand{\fr}[1]{#1}
  \newcommand{\en}[1]{}
\fi

%%%%%%%%%%%%%%%%%%%%%%%%%%%%%%%%%%%%%%%%%%%%%%%%%%%%%%%%%%

\firstname{Robin}
\familyname{Moussu}
\title{\fr{Développeur logiciel}\en{Software engineer}}
\address{17 rue Pasteur}{38400 Saint-Martin d'Heres}
\phone[mobile]{06 95 07 44 23}
\email{robin.moussu@gmail.com}
\extrainfo{
  \linkedinsocialsymbol \githubsocialsymbol robinmoussu
  \makenewline
  \arabic{age} \fr{ans}\en{years old}
}
\photo{photo.jpg}

\begin{document}

\maketitle

\vspace{\fill}

\hspace{1.5cm} \fr{Je viens de terminer ma formation d'ingénieur en informatique, et je suis actuellement à la recherche d'un emploi.}
\en{I just graduated from Grenoble INP, a computer science engineering school, and I am currently searching for a job.}

\vspace{\fill}

\section{\fr{Compétences}\en{Competences}}

\cvitem{\fr{Langages}\en{Languages}}{\xx{\cpp} (\fr{mon langage de prédilection}\en{my favorite language}), \xx{C}, \xx{ruby}, ada, asm, vhdl}
\cvitem{\fr{Outils}\en{Utils}}{\xx{git}, gcc, clang, Doxygen} 
%\cvitem{Embedded}{\pic, \stm, arduino, rasberry pi}
\cvitem{\fr{Conception}\en{Methods}}{Design patterns, \uml{}} 
\cvitem{\fr{Autre}\en{Other}}{\xx{Linux}, Latex, \ldots}

\vspace{1em}

\cvitemwithcomment{\fr{Anglais}\en{English}}{\fr{Compétences techniques, niveau C1}\en{Technical knowledge, C1 level}}{\hfill{}Bulats: 79 (2014), Toeic: 740 (2013)}{}

\section{Formation}

\cvformation{2013--2016}{Grenoble INP}{Grenoble}{Phelma \fr{puis}\en{then} Ensimag}{}{\fr{Spécialité en Systemes et Logiciels Embarqués}\en{Specialization in system and embedded software}}
\cvformation{2011--2013}{DUT GEII}{Grenoble}{IUT 1}{}{\fr{Génie électronique et informatique industrielle}\en{Electronic engineering and industrial data processing}}
\cvformation{2010--2011}{Prépa intégré}{Valence}{Esisar}{}{\fr{Électronique, informatique, système embarqué et réseaux}\en{Electronics, computer science, embedded system and network}}
\cvformation{2010}{Baccalauréat S}{Gap}{Lycée Dominique Villars (\fr{Mention bien}\en{with honors})}{}{}

\section{\fr{Stages}\en{Internships}}

\cvstage{2016}{}{MathWorks (\fr{logiciel Polyspace}\en{on Polyspace software})}{Montbonnot, Isère}{6 \fr{mois}\en{month}}{\fr{Analyse d'alias \& séparation de contextes (C++ et SML)}\en{Alias analysis and context splitting (C++ and SML)}}
\cvstage{2015}{}{Start Me Up}{Meylan, Isère}{3 \fr{mois}\en{month}}{\fr{Streaming video (C++)}\en{Video streaming (C++)}}
\cvstage{2014}{}{Kalray}{Montbonnot, Isère}{3 \fr{mois}\en{month}}{\fr{Description d'architecture processeur many-core}\en{Description of many-core processor architecture} (ruby)}
\cvstage{2013}{}{Max Technologies}{Brossard, \textbf{Canada}}{3 \fr{mois}\en{month}}{\fr{\textbf{Portage de code} du langage C vers le langage \labview}\en{Code re-writing from C to \labview}}

\section{\fr{Expériences personnelles}\en{Personal experiences}}

\cvcomp{2015}{\fr{Contribution à Firefox}\en{Contribution to Firefox}}{\fr{projet scolaire}\en{scholar project}}{}{\fr{par groupe de 5}\en{by group of 5 students}}{\fr{1 mois}\en{1 month}} %{Correction de bugs et amélioration de la partie audio et graphique de Firefox}
\cvcomp{2015}{\fr{Conception d'un compilateur}\en{Conception of a compiler}}{\fr{projet scolaire}\en{scholar project}}{}{\fr{par groupe de 5}\en{by group of 5 students}}{\fr{1 mois}\en{1 month}} %{Implémentation en java d'un compilateur et de sa librairie mathématique pour le Déca, un langage proche du java}
\cvcomptwo{2011-2015}{\fr{Responsable de classe}\en{Class representative}}{Ensimag, Phelma \fr{et}\en{and} IUT}{Grenoble}{}{\fr{Participation au conseil d’administration (durant mon IUT)}\en{Participation to the board of directors during my IUT}}

\vspace{1em}

%\cvitemwithcomment{Projet}{Conception d'un clavier ergonomique}{réalisation matérielle et logicielle}
\cvcomptwo{\fr{Robotique}\en{Robotics}}{\fr{Coupe de France de Robotique}\en{French cup of Robotics}}{}{}{}{
  {\small
    \fr{
      Phelma: Robotronik
              (2015~: 103\textsuperscript{e},
              2014~: 91\textsuperscript{e}),
      club de robotique de l'Esisar (2011)
    }
    \en{
      Phelma: Robotronik
              (2015:~103\textsuperscript{th},
              2014:~91\textsuperscript{th}),
      robotics club of Esisar (2011)
    }
  }
  \break
  {
  \textbf{\fr{Coupe inter-IUT}\en{inter-IUT cups}}, Vierzon, 2012, IUT GEII \fr{de}\en{of} Grenoble, \fr{demi-finale}\en{semi-final}
  }
}
%\cvitemwithcomment{Loisirs}{Jeu de société, vélo, escalade}{}

\end{document}
